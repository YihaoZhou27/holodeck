\usepackage{xifthen}

\newcommand{\thard}{\tau}
\newcommand{\thardf}{\tau_f}
\newcommand{\holodeck}{\texttt{holodeck}}
\newcommand{\python}{\texttt{python}}
\newcommand{\mmbulge}{{$M_\textrm{BH}$--$M_\textrm{bulge}$}}

\definecolor{purple1}{rgb}{0.6, 0.0, 0.8}

% \newcommand{\todo}[1]{{\textbf{\color{purple1}TODO: #1}}}
\newcommand{\todo}[1]{{\textbf{\color{purple1}\{#1\}}}}
\newcommand{\NOTE}[1]{\noindent\textbf{\color{red}!!#1!!}}
\newcommand{\note}[1]{{\color{purple1}#1}}
\newcommand*{\needcite}[1]{
    \ifthenelse{\equal{#1}{}}{
        {\color{red}[???]}
    }{
        {\color{red} [#1]}
    }
}

\newcommand{\tr}[1]{\textrm{#1}}
\newcommand{\trt}[1]{\textrm{\tiny{#1}}}
\newcommand{\trf}[1]{\textrm{\footnotesize{#1}}}


% ==============================================================================
% ====    Units and General Astronomy Symbols
% ==============================================================================

% ---- Units

\newcommand{\msol}{\tr{M}_{\odot}}
\newcommand{\rsol}{\tr{R}_{\odot}}
\newcommand{\pc}{\mathrm{pc}}
\newcommand{\yr}{\mathrm{yr}}        % yr in math-mode
\newcommand{\sfluxunits}{\textrm{ erg/s/Hz/cm}^2}
\newcommand{\invmpccubed}{\textrm{Mpc}^{-3}}
\newcommand{\as}{\textrm{arcsec}}
\newcommand{\pyr}{\textrm{yr}^{-1}}

% ---- Symbols

\newcommand{\poisson}{\mathcal{P}}

% \newcommand{\ndens}{n_c}
\newcommand{\ndens}{\eta_c}
% \newcommand{\number}{N}

\newcommand{\ayr}{A_{\trt{yr}^{-1}}}        % GWB amplitude normalization at 1/yr
\newcommand{\mbh}{M_\bullet}
\newcommand{\mstar}{M_\star}
\newcommand{\rbh}{R_\bullet}
\newcommand{\rstar}{R_\star}

\newcommand{\bigt}{\ensuremath{\uptau}}

\newcommand{\hc}{h_\trt{c}}
\newcommand{\hs}{h_\trt{s}}
\newcommand{\hsn}{h_\tr{s,n}}
\newcommand{\hscirc}{h_\trt{s,circ}}

\newcommand{\mdot}{\dot{M}}
\newcommand{\mdotedd}{\dot{M}_\trt{Edd}}
\newcommand{\ledd}{L_\trt{Edd}}    % Eddington luminosity
\newcommand{\lacc}{L_\trt{acc}}    % accretion luminosity
\newcommand{\radeff}{\varepsilon_\trt{rad}}

\newcommand{\fedd}{f_\trt{Edd}}    % Mdot-eddington factor for limiting
\newcommand{\mchirp}{\mathcal{M}}     % Chirp-mass

\newcommand{\distlum}{d_\trt{L}}   % luminosity distance
\newcommand{\distcom}{d_c}   % comoving distance

\newcommand{\fobs}{f_\tr{obs}}   % observer-frame orbital-frequency
\newcommand{\fobsgw}{f_\tr{GW,obs}}   % observer-frame GW-frequency
\newcommand{\frst}{f_\tr{rst}}   % rest-frame orbital-frequency
\newcommand{\frstgw}{f_\tr{GW,rst}}   % rest-frame GW-frequency
\newcommand{\frstorb}{f_{\trt{orb},r}}
\newcommand{\fobsorb}{f_{\trt{orb},o}}
\newcommand{\frstn}{f_\trt{n,r}}
\newcommand{\fobsn}{f_\trt{n,o}}

\newcommand\erfc[1]{\mathrm{erfc}\left(#1\right)}
\newcommand\erfcinv[1]{\mathrm{erfc}^{-1}\left(#1\right)}

\newcommand{\lgw}{L_\trt{GW}}
\newcommand{\lgwn}{L_{\trt{GW},n}}
\newcommand{\lgwc}{L_\trt{GW,circ}}
\newcommand{\egw}{\varepsilon_\trt{GW}}   % energy in GW




% ==============================================================================
% ====    General Math Stuff
% ==============================================================================

\newcommand{\E}[1]{\times\nobreak10^{#1}}
\newcommand{\ls}{\lesssim}
\newcommand{\gs}{\gtrsim}
\newcommand{\logten}[1]{\log_{10}\!\lr{#1}}

\newcommand{\sinpar}[2][]{\sin^{#1}\!\lr{#2}}
\newcommand{\cospar}[2][]{\cos^{#1}\!\lr{#2}}

% left-right parentheses
\newcommand{\lr}[2][]{
    \ifthenelse{\equal{#1}{}}{
        % omitted
        {\left(#2\right)}
    }{
        % given
        {\left(#2\right)}^{#1}
    }
}

% left-right square-bracket
\newcommand{\lrs}[2][]{
    \ifthenelse{\equal{#1}{}}{
        % omitted
        {\left[#2\right]}
    }{
        % given
        {\left[#2\right]}^{#1}
    }
}

% left-right-tight (parenthesis
\newcommand{\lrt}[1]{{\left(\!#1\!\right)}}
% left-right-tight
\newcommand{\lrst}[1]{{\left[\!#1\!\right]}}

% optional first argument for exponent
%    i.e. `\scale{A}{B} = (A/B)` or `\scale[2]{A}{B} = (A/B)^2`
\newcommand{\scale}[3][]{
    \ifthenelse{\equal{#1}{}}{
        % omitted
        \lr{ \frac{#2}{#3} }
    }{
        % given
        {\lr[#1]{ \frac{#2}{#3} }}
    }
}

% ==============================================================================
% ====    Logistical / Auxiliary Commands
% ==============================================================================

\newcommand{\figref}[1]{Fig.~\ref{#1}}
\newcommand{\secref}[1]{\textsection\ref{#1}}
\newcommand{\refeq}[1]{{Eq.~\ref{#1}}}
\newcommand{\tabref}[1]{{Table~\ref{#1}}}
\newcommand{\fnm}[1]{\footnotemark[#1]}
\newcommand{\fnt}[2]{\footnotemark[#1]{#2}}
\newcommand{\mc}[2]{\multicolumn{#1}{c}{#2}}
\newcommand{\mr}[2]{\multirow{#1}{*}{#2}}
